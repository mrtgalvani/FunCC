\documentclass[a4paper]{book}
\usepackage[times,inconsolata,hyper]{Rd}
\usepackage{makeidx}
\usepackage[utf8]{inputenc} % @SET ENCODING@
% \usepackage{graphicx} % @USE GRAPHICX@
\makeindex{}
\begin{document}
\chapter*{}
\begin{center}
{\textbf{\huge Package `FunCC'}}
\par\bigskip{\large \today}
\end{center}
\begin{description}
\raggedright{}
\inputencoding{utf8}
\item[Title]\AsIs{Functional Cheng and Church Bi-clustering}
\item[Version]\AsIs{0.0.0.9000}
\item[Description]\AsIs{The FunCC package allows to apply the funCC algorithm to simultaneously cluster the rows and the columns of a data matrix whose inputs are functions.}
\item[Depends]\AsIs{R (>= 3.5.1)}
\item[License]\AsIs{What license is it under?}
\item[Encoding]\AsIs{UTF-8}
\item[LazyData]\AsIs{true}
\item[RoxygenNote]\AsIs{7.0.2}
\item[Imports]\AsIs{narray,
biclust,
reshape,
RColorBrewer,
ggplot2}
\end{description}
\Rdcontents{\R{} topics documented:}
\inputencoding{utf8}
\HeaderA{find\_best\_delta}{Functional Cheng and Church Algorithm varying the delta value}{find.Rul.best.Rul.delta}
%
\begin{Description}\relax
The find\_best\_delta function evaluate the results of FunCC algorithm in terms of total H-score value, the number of obtained bi-clusters and the number of not assigned elements when varying the delta value
\end{Description}
%
\begin{Usage}
\begin{verbatim}
find_best_delta(
  fun_mat,
  delta_min,
  delta_max,
  num_delta = 10,
  template.type = "mean",
  theta = 1.5,
  number = 100,
  alpha = 0,
  beta = 0,
  const_alpha = F,
  const_beta = F,
  shift.alignement = F,
  shift.max = 0.1,
  max.iter.align = 100
)
\end{verbatim}
\end{Usage}
%
\begin{Arguments}
\begin{ldescription}
\item[\code{fun\_mat}] The data array (n x m x T) where each entry corresponds to the measure of one observation i, i=1,...,n, for a functional variable m, m=1,...,p, at point t, t=1,...,T

\item[\code{delta\_min}] scalar: Maximum value of the maximum of accepted score, should be a real value > 0

\item[\code{num\_delta}] integer: number of delta to be evaluated between delta\_min and delta\_max

\item[\code{template.type}] character: type of template required. If template.type='mean' the template is evaluated as the average function, if template.type='medoid' the template is evaluated as the medoid function.

\item[\code{theta}] scalar: Scaling factor should be a real value > 1

\item[\code{number}] integer: Maximum number of iterations

\item[\code{alpha}] binary: if alpha=1 row shift is allowed, if alpha=0 row shift is avoided

\item[\code{beta}] binary: if beta=1 row shift is allowed, if beta=0 row shift is avoided

\item[\code{const\_alpha}] logicol: indicates if row shift is contrained as constant

\item[\code{shift.alignement}] logicol: If shift.alignement=True the shift aligment is performed, if shift.alignement=False no alignment is performed

\item[\code{shift.max}] scalar: shift.max controls the maximal allowed shift, at each iteration, in the alignment procedure with respect to the range of curve domains. t.max must be such that 0<shift.max<1

\item[\code{max.iter.align}] integer: maximum number of iteration in the alignment procedure

\item[\code{const\_b}] logicol: indicates if col shift is contrained as constant
\end{ldescription}
\end{Arguments}
%
\begin{Value}
a dataframe containing for each evaluated delta: Htot\_sum (the sum of totale H-score), num\_clust (the number of found Bi-clusters), not\_assigned (the number of not assigned elements)
\end{Value}
\inputencoding{utf8}
\HeaderA{funCCdata}{Simulated data}{funCCdata}
\keyword{datasets}{funCCdata}
%
\begin{Description}\relax
funCC.data is a functional dataset displaying block structure
\end{Description}
%
\begin{Usage}
\begin{verbatim}
data(funCCdata)
\end{verbatim}
\end{Usage}
%
\begin{Format}
An object of class \code{array} of dimension 30 x 7 x 240.
\end{Format}
%
\begin{Examples}
\begin{ExampleCode}
data(funCCdata)

\end{ExampleCode}
\end{Examples}
\inputencoding{utf8}
\HeaderA{funcc\_biclust}{Functional Cheng and Church algorithm}{funcc.Rul.biclust}
%
\begin{Description}\relax
The funCC algorithm allows to simultaneously cluster the rows and the columns of a data matrix where each entry of the matrix is a function or a time series
\end{Description}
%
\begin{Usage}
\begin{verbatim}
funcc_biclust(
  fun_mat,
  delta,
  theta = 1,
  template.type = "mean",
  number = 100,
  alpha = 0,
  beta = 0,
  const_alpha = F,
  const_beta = F,
  shift.alignement = F,
  shift.max = 0.1,
  max.iter.align = 100
)
\end{verbatim}
\end{Usage}
%
\begin{Arguments}
\begin{ldescription}
\item[\code{fun\_mat}] The data array (n x m x T) where each entry corresponds to the measure of one observation i, i=1,...,n, for a functional variable m, m=1,...,p, at point t, t=1,...,T

\item[\code{delta}] scalar: Maximum of accepted score, should be a real value > 0

\item[\code{theta}] scalar: Scaling factor should be a real value > 1

\item[\code{template.type}] character: type of template required. If template.type='mean' the template is evaluated as the average function, if template.type='medoid' the template is evaluated as the medoid function.

\item[\code{number}] integer: Maximum number of iteration

\item[\code{alpha}] binary: if alpha=1 row shift is allowed, if alpha=0 row shift is avoided

\item[\code{beta}] binary: if beta=1 row shift is allowed, if beta=0 row shift is avoided

\item[\code{const\_alpha}] logicol: Indicates if row shift is contrained as constant.

\item[\code{shift.alignement}] logicol: If shift.alignement=True the shift aligment is performed, if shift.alignement=False no alignment is performed

\item[\code{shift.max}] scalar: shift.max controls the maximal allowed shift, at each iteration, in the alignment procedure with respect to the range of curve domains. t.max must be such that 0<shift.max<1

\item[\code{max.iter.align}] integer: maximum number of iteration in the alignment procedure

\item[\code{const\_b}] logicol: Indicates if col shift is contrained as constant.
\end{ldescription}
\end{Arguments}
%
\begin{Value}
a list of two elements containing respectively the Biclustresults and a dataframe containing the parameters setting of the algorithm
\end{Value}
\inputencoding{utf8}
\HeaderA{funcc\_show\_bicluster\_coverage}{plotting coverage of each bi-cluster}{funcc.Rul.show.Rul.bicluster.Rul.coverage}
%
\begin{Description}\relax
funcc\_show\_bicluster\_coverage graphically shows the coverage of each bi-cluster in terms of percentage of included functions
\end{Description}
%
\begin{Usage}
\begin{verbatim}
funcc_show_bicluster_coverage(
  fun_mat,
  res_input,
  not_assigned = T,
  max_coverage = 1
)
\end{verbatim}
\end{Usage}
%
\begin{Arguments}
\begin{ldescription}
\item[\code{fun\_mat}] The data array (n x m x T) where each entry corresponds to the measure of one observation i, i=1,...,n, for a functional variable m, m=1,...,p, at point t, t=1,...,T

\item[\code{res\_input}] An object produced by the funcc\_biclust function

\item[\code{not\_assigned}] logicol: if true also the cluster of not assigned elements is included

\item[\code{max\_coverage}] scalar: percentage of maximum cumulative coverage to be shown
\end{ldescription}
\end{Arguments}
%
\begin{Value}
a figure representing for each bi-cluster the coverage in terms of percentage of included functions
\end{Value}
\inputencoding{utf8}
\HeaderA{funcc\_show\_bicluster\_dimension}{plotting dimensions of each bi-cluster}{funcc.Rul.show.Rul.bicluster.Rul.dimension}
%
\begin{Description}\relax
funcc\_show\_bicluster\_dimension graphically shows the dimensions of each bi-cluster (i.e. number of rows and columns)
\end{Description}
%
\begin{Usage}
\begin{verbatim}
funcc_show_bicluster_dimension(fun_mat, res_input)
\end{verbatim}
\end{Usage}
%
\begin{Arguments}
\begin{ldescription}
\item[\code{fun\_mat}] The data array (n x m x T) where each entry corresponds to the measure of one observation i, i=1,...,n, for a functional variable m, m=1,...,p, at point t, t=1,...,T

\item[\code{res\_input}] An object produced by the funcc\_biclust function
\end{ldescription}
\end{Arguments}
%
\begin{Value}
a figure representing the dimensions of each bi-cluster (i.e. number of rows and columns)
\end{Value}
\inputencoding{utf8}
\HeaderA{funcc\_show\_block\_matrix}{Plotting co-clustering results of funCC on the data matrix}{funcc.Rul.show.Rul.block.Rul.matrix}
%
\begin{Description}\relax
funcc\_show\_block\_matrix graphically shows the bi-clusters positions in the original data matrix
\end{Description}
%
\begin{Usage}
\begin{verbatim}
funcc_show_block_matrix(fun_mat, res_input)
\end{verbatim}
\end{Usage}
%
\begin{Arguments}
\begin{ldescription}
\item[\code{fun\_mat}] The data array (n x m x T) where each entry corresponds to the measure of one observation i, i=1,...,n, for a functional variable m, m=1,...,p, at point t, t=1,...,T

\item[\code{res\_input}] An object produced by the funcc\_biclust function
\end{ldescription}
\end{Arguments}
%
\begin{Value}
a figure representing the bi-clusters positions in the original data matrix
\end{Value}
\inputencoding{utf8}
\HeaderA{funcc\_show\_results}{Plotting co-clustering results of funCC}{funcc.Rul.show.Rul.results}
%
\begin{Description}\relax
funcc\_show\_results graphically shows the results of the bi-clustering
\end{Description}
%
\begin{Usage}
\begin{verbatim}
funcc_show_results(fun_mat, res_input, only.mean = F, aligned = F, warping = F)
\end{verbatim}
\end{Usage}
%
\begin{Arguments}
\begin{ldescription}
\item[\code{fun\_mat}] The data array (n x m x T) where each entry corresponds to the measure of one observation i, i=1,...,n, for a functional variable m, m=1,...,p, at point t, t=1,...,T

\item[\code{res\_input}] An object produced by the funcc\_biclust function

\item[\code{only.mean}] logicol: if True only the template functions for each bi-cluster is displayed

\item[\code{aligned}] logicol: if True the alignemd functions are displayed

\item[\code{warping}] logicol: if True also a figure representing the warping functions are displayed
\end{ldescription}
\end{Arguments}
%
\begin{Value}
a figure representing each bi-cluster in terms of functions contained in it or templates
\end{Value}
\printindex{}
\end{document}
